%!TEX program = xelatex
\documentclass[UTF8]{ctexart}
\usepackage{titlesec}
\usepackage{listings}
\usepackage{xcolor}
\usepackage{graphicx}
\usepackage{geometry}
\usepackage{subfigure}
\usepackage{fancyhdr}
\usepackage{amsfonts,amssymb}
\usepackage{amsmath}
\usepackage{latexsym}
\usepackage{algorithm2e}
\usepackage{tikz}
\usepackage{bm}
\usepackage{placeins}
\geometry{left = 3cm, right = 3cm, top = 3.0cm, bottom = 2.5cm}
\pagestyle{fancy}
\lhead{}
\chead{}
\rhead{}
\lfoot{}
\cfoot{\thepage}
\rfoot{}
\definecolor{CPPLight}  {HTML} {686868}
\definecolor{CPPSteel}  {HTML} {888888}
\definecolor{CPPDark}   {HTML} {262626}
\definecolor{CPPBlue}   {HTML} {4172A3}
\definecolor{CPPGreen}  {HTML} {487818}
\definecolor{CPPBrown}  {HTML} {A07040}
\definecolor{CPPRed}    {HTML} {AD4D3A}
\definecolor{CPPViolet} {HTML} {7040A0}
\definecolor{CPPGray}  {HTML} {B8B8B8}
\lstset{
    columns=fixed,
    %numbers=left,
    frame=single,
    %numberstyle=\scriptsize,
    basicstyle=\ttfamily\small,
    keywordstyle=\color[RGB]{40,40,255},
    numberstyle=\footnotesize\color{darkgray},
    commentstyle=\it\color[RGB]{0,96,96},
    stringstyle=\rmfamily\slshape\color[RGB]{128,0,0},
    showstringspaces=false,
    language=c++,
    morekeywords={alignas,continute,friend,register,true,alignof,decltype,goto,
    reinterpret_cast,try,asm,defult,if,return,typedef,auto,delete,inline,short,
    typeid,bool,do,int,signed,typename,break,double,long,sizeof,union,case,
    dynamic_cast,mutable,static,unsigned,catch,else,namespace,static_assert,using,
    char,enum,new,static_cast,virtual,char16_t,char32_t,explict,noexcept,struct,
    void,export,nullptr,switch,volatile,class,extern,operator,template,wchar_t,
    const,false,private,this,while,constexpr,float,protected,thread_local,
    const_cast,for,public,throw,std},
    emph={map,set,multimap,multiset,unordered_map,unordered_set,
    unordered_multiset,unordered_multimap,vector,string,list,deque,
    array,stack,forwared_list,iostream,memory,shared_ptr,unique_ptr,
    random,bitset,ostream,istream,cout,cin,endl,move,default_random_engine,
    uniform_int_distribution,iterator,algorithm,functional,bing,numeric,},
    emphstyle=\color{CPPViolet}
}
\newcommand\scalemath[2]{\scalebox{#1}{\mbox{\ensuremath{\displaystyle #2}}}}
\title{Python大作业:Text Compressor}
\author{
{\small 施晓钰, 516030910567}
}
\begin{document}
\maketitle

\section{Text Compressor简介}
此压缩器可通过重新编码对英文文本文档进行压缩和解压操作。\\
\indent 压缩操作:点击“选择文件”可在本地选择一文本,压缩成一个二进制文件和一个对应的字典存入通过“选择输出路径”指定的文件夹,并输出压缩率。其中二进制文件命名为compressed.out,字典命名为dict.txt。\\
\indent 解压操作:点击“选择文件”选择本地的二进制文件,点击“选择字典”选择与该二进制文件对应的字典,将此二进制文件解压后存入通过“选择输出路径”指定的文件夹。此过程可能较慢。\\
\indent 若使用command line版本则直接运行程序,按提示进行输入即可。\\
\indent 注意:此文本压缩器只适用于英文文档。

\section{Text Compressor实现}
压缩部分:
\begin{itemize}
\item 生成字典:首先按字符遍历该文档,对每个字符在文档中的出现次数进行记数,并以此构建哈夫曼树。这里采用了sorted函数,将字符由小到大排成一个队列,每次取出最小的两个字符作为新节点的儿子节点,并将它们的值之和作为新节点的值,在队列中删除这两个字符并加入新的节点,重新排序,重复以上过程直到队列中只剩下一个节点。再由根对哈夫曼树进行遍历,并对每个节点进行赋值作为它的新编码。对于每一个节点而言,它的左儿子的编码是他的编码末端加上‘0’,右儿子则加上‘1’。
\item 生成二进制文件:遍历整个文档,将每一个字符换成对应的哈夫曼编码,再将这个长长的01串进行以八位为一单位的切割并存成一个char,将这个转化后的文件存为二进制文件。
\end{itemize}

\indent 解压部分:

\begin{itemize}
\item 字典导入:通过上传的字典文件生成相对应的哈夫曼树。对于每一个字符,从根开始建立新的节点,最终将每一个字符都加到哈夫曼树的叶子节点上。\\
\item 文件解码:将每个字符再转化为八位二进制编码,利用这个01串对建立好的字典树进行查询。由于构建了哈夫曼树,所有字符对应的节点都是叶子节点,没有字符的编码是其他字符编码的前几位,因此只需直接依据该01串进行查询直到找到一个字符,再返回树的根节点进行下一轮查询即可。翻译出的文件即为原来未压缩的文件。
\end{itemize}

压缩原理:利用了哈夫曼树,可以将使用频率较大的字符存成最低至三位的编码,使用频率小的字符可能会对应相对较长的编码,但总的来说相比于原来的八位占用了更小的空间。

\section{一些经历}
作者是计算机系的学生,但之前没有接触过python,所以对python的一些函数不是很了解,所以一开始想利用heapq的堆进行字符计数后的排序,但最后发现它只是一个堆没有进行排序。于是就改用了sorted函数。\\
\indent 对于存成二进制文件这件事之前一直不是很理解,把文本变成了01串之后就进行了储存,结果发现比原来的文件还大,花了一段时间弄明白切割和转码的操作。\\
\indent 因为更倾向于用commandline操作,所以ui画面没有进行美化。(但commandline需要手动输入文件路径也挺麻烦的。)
\end{document}















